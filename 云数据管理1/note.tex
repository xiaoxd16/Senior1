\documentclass{article}
\usepackage{CJKutf8}
\usepackage{amsmath}

\begin{document}
\begin{CJK}{UTF8}{gbsn}
	\section*{ACID}
	A:原子性\\
	C:一致性\\
	I:隔离性\\
	D:耐久性\\
	\section*{范式}
	\subsection*{第一范式}
	二维数据表,是第一范式才是数据库\\
	\subsection*{第二范式}
	第一范式+主键+其他列完全依赖主键\\
	\subsection*{第三范式}
	第二范式+属性不依赖于非主键(不存在非主属性的传递函数依赖)\\
	\subsection*{BC范式}
	第三范式+不存在任何字段对任一候选关键字段的传递函数依赖\\
	
	\section*{DDBMS}
	查询引擎\\
	事务管理\\
	数据集成/多数据\\
	复制/并行\\
	\section*{DBMS优点}
	独立:与上层应用分离\\
	缓存管理:性能优势(即使有数据丢失也可以通过日志恢复)\\
	ACID\\
	辅助函数\\
	\section*{DDBMS优点}
	数据独立性:\\
	高可靠/高可用:可靠指的是出错少,可用指的是出错后可以尽快恢复\\
	高性能:\\
	可伸缩:\\
	\section*{DDBMS缺点}
	没有标准\\
	复杂\\
	管理困难\\
	安全性\\
	\section*{DDBMS Architectures}
	\subsection*{ANSI/SPARC Architecture}
	用户-外模式-概念模式-内模式\\
	\subsection*{C/S DDBMS}
	SQL接口、程序接口、缓存-网络-目录、查询分解\\
	客户端-应用服务器-数据服务器-数据库\\
	\subsection*{P2P DDBMS}
	外模式-GCS(Global Conceptual Schema)-LCS(Local Conceptual Schema)-LIS(Local Internal Conceptual)\\
	\subsection*{MDBMS Architectures}
	MDBMS
	\section*{模式、内模式与外模式}
	\subsection*{模式}
	定义:也称逻辑模式,是数据库中全体数据的逻辑结构和特征的描述,是所有用户的公共数据视图。\\
	一个数据库只有一个模式\\
	是数据库数据在逻辑级上的视图\\
	数据库模式以某一种数据模型为基础\\
	定义模式时不仅要定义数据的逻辑结构(如数据记录由哪些数据项构成,数据项的名字、类型、取值范围等),而且要定义与数据有关的安全性、完整性要求,定义这些数据之间的联系\\
	\subsection*{内模式}
	定义:也称子模式(Subschema)或用户模式,是数据库用户(包括应用程序员和最终用户)能够看见和使用的局部数据的逻辑结构和特征的描述,是数据库用户的数据视图,是与某一应用有关的数据的逻辑表示。\\
	一个数据库可以有多个外模式\\
	外模式就是用户视图\\
	外模式是保证数据安全性的一个有力措施\\
	\subsection*{外模式}
	定义:也称存储模式(Storage Schema),它是数据物理结构和存储方式的描述,是数据在数据库内部的表示方式(例如,记录的存储方式是顺序存储、按照B树结构存储还是按hash方法存储;索引按照什么方式组织;数据是否压缩存储,是否加密;数据的存储记录结构有何规定)。\\
	一个数据库只有一个内模式\\
	一个表可能由多个文件组成,如:数据文件、索引文件\\
	它是数据库管理系统(DBMS)对数据库中数据进行有效组织和管理的方法,可以减少数据冗余,实现数据共享,还可以减少数据冗余,实现数据共享\\
	\section*{自顶向下设计(Top-down Design)}
	\subsection*{数据分解}
	水平分解:表结构不变,数据分成不同数据集\\
	垂直分解:分解成多个子表\\
	优点:\\
	查询快\\
	缺点:\\
	性能的影响\\
	完整性保证\\
	分解粒度大小\\
	更新慢\\
	\subsubsection*{如何确保正确性}
	完备性\\
	不相交性\\
	可重构性\\
	
	
	\section*{生命周期}
	\subsection*{从数据到大数据}
	大是相对的,和数据量,难度,处理时间等相关\\
	无政府主义抬头\\
	价值密度稀疏\\
	OLTP在线事务$\rightarrow$OLAP在线分析$\rightarrow$BI商务智能\\
	大致流程:获取数据$\rightarrow$抽取清洗$\rightarrow$集成聚合$\rightarrow$分析建模$\rightarrow$解释展示\\
	\subsection*{大数据技术体系}
	
	
	
	\section*{分表、分区、分片与分库}
	\subsection*{Allocation Alternatives}
	优势:访问快,可靠性高\\
	劣势:更新慢,易出错,必须同时更新所有\\
	\subsection*{为什么分割}
	查询效率\\
	可靠性和可用性\\
	安全性\\
	\subsection*{水平分割(Horizontal Fragmentation)}
	根据特定Property进行分割\\
	用于分割的简单谓词集合$P_r$应当是最小而且完备的//
	EX:$\sigma_{BUDGET<1000}(PROJ)$\\
	如何获取最小完备的简单谓词集合\\
	主要是把重复的给消去,比如$B<100$和$B\geq 100$\\
	\subsubsection*{Primary Horizontal Fragmentation}
	$R_i=\sigma_{F_i}(R),1\leq i\leq w$\\
	其中,$F_i$应当是最小项谓词\\	
	\subsubsection*{Derived Horizontal Fragmentation}
	
	
	\subsection*{垂直分割(Vertical Fragmentation)}
	需要复制主键,按属性切割\\
	\subsection*{混合切割(Mixed Fragmentation)}
	综合上面二者进行切割\\
	\subsection*{如何确保分割的正确性}
	完备性:每个item必须至少属于一张子表\\
	不相交性:一个数据不能属于两个分割后的表\\
	重构性:可以重构出原表\\
	\subsection*{Data Fragmentation Design}
	\subsubsection*{简单谓词(Simple Predicate)}
	$P_j:A_i\Theta\ value\ where\ \Theta\in\lbrace=,<,\leq,>,\geq,\neq \rbrace$
	Ex:NAME $\neq$ "MAIN"\\
	\subsubsection*{最小项谓词(Minterm Predicate)}
	是简单谓词的结合,并且任取两项,要么相等,要么相反\\
	Ex:\\
	m1:NAME="A" $\land$ BUDGET $<$ 2000\\
	m2:NOT(NAME="A") $\land$ BUDGET $<$ 2000\\
	m3:NAME="A" $\land$ NOT(BUDGET $<$ 2000)\\
	m4:NOT(NAME="A") $\land$ NOT(BUDGET $<$ 2000)\\
	
	
	\section*{软件栈}
	
	
	\subsection*{BASE}
	Basically\\
	Available\\
	Soft-state\\
	Eventual consistency\\
	
	\subsection*{Hadoop生态圈}
	利用MapReduce作为核心的生态圈\\
	缺陷:频繁的IO\\
	
	
	\section*{GFS/HDFS}
	分布式文件系统:GFS/HDFS\\
	NoSQL:HBASE/Cassandra/MongoDB\\
	\subsection*{NoSQL分类}
	Key/Value:\\
	Schemaless:语义结构不够强,没有传统的ACID特性\\
	
	\subsection*{CAP}
	C:Consistency一致性\\
	A:Availability可用性\\
	P:Partition Tolerance\\
	\subsection*{Sparding(文件分块)}
	问题:容错性极地,但是只要有一块坏掉了,原始数据无法恢复\\
	使用Replication(副本机制):安全(相当于多一个备份),性能更高\\
	出现的问题:更新时一致性问题\\
	解决方法:Master/Slave机制\\
	写的操作由Master进行,读由slave进行\\
	仍然存在的问题:\\
	Master写入多个Slave仍然需要时间,可能存在延迟\\
	Master工作量大\\
	单点故障问题\\
	解决方法:P2P\\
	1.用户必须等所有更新完毕才能离开(保证数据一致性,但是速度慢,并且网络故障后会一直等待)\\
	2.用户只需要更新一个节点,剩下的自行完成(无法保证数据一致性,可以通过全部读取选择最新来完成,但是仍然会受到网络故障的影响)\\
	
	\subsection*{GFS}
	\subsubsection*{假设与目标}
	流数据读写:主要用于程序处理批量数据,而非与用户的交互或随机读写,所以主要是追加写\\
	文件尺寸大\\
	\subsubsection*{设计思路}
	1.分块:一个Chunk64M\\
	2.通过冗余提高可靠性(多个副本)\\
	3.通过单个master协调数据访问、元数据存储\\
	\subsubsection*{问题}
	单点故障
	性能瓶颈
	\subsubsection*{解决方案}
	尽量减少Master参与程度\\
	不使用Master读取数据,仅用于保存元数据\\
	客户端缓存元数据\\
	使用大尺寸数据块64M\\
	\subsubsection*{Master的功能}
	存储元数据\\
	文件系统目录管理与加锁\\
	与ChunkServer进行周期性通信\\
	数据块创建,复制与负载均衡\\
	垃圾回收\\
	陈旧数据快删除\\
	\subsubsection*{元数据}
	只有三个类型的元数据:\\
	1.文件和块的命名空间\\
	2.从文件到块的映射\\
	3.每个块的副本位置\\
	\subsubsection*{GFS的特点}
	采用中心服务器模式\\
	
	
	\subsubsection*{GFS的容错机制}
	三类元数据,前两类可以使用log恢复,最后通过备份恢复\\
\end{CJK}
\end{document}